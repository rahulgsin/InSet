% Generated by Sphinx.
\def\sphinxdocclass{report}
\documentclass[letterpaper,10pt,english]{sphinxmanual}
\usepackage[utf8]{inputenc}
\DeclareUnicodeCharacter{00A0}{\nobreakspace}
\usepackage{cmap}
\usepackage[T1]{fontenc}
\usepackage{babel}
\usepackage{times}
\usepackage[Bjarne]{fncychap}
\usepackage{longtable}
\usepackage{sphinx}
\usepackage{multirow}

\addto\captionsenglish{\renewcommand{\figurename}{Fig. }}
\addto\captionsenglish{\renewcommand{\tablename}{Table }}
\floatname{literal-block}{Listing }



\title{InSet Documentation}
\date{July 15, 2015}
\release{1}
\author{Rahul Singh}
\newcommand{\sphinxlogo}{}
\renewcommand{\releasename}{Release}
\makeindex

\makeatletter
\def\PYG@reset{\let\PYG@it=\relax \let\PYG@bf=\relax%
    \let\PYG@ul=\relax \let\PYG@tc=\relax%
    \let\PYG@bc=\relax \let\PYG@ff=\relax}
\def\PYG@tok#1{\csname PYG@tok@#1\endcsname}
\def\PYG@toks#1+{\ifx\relax#1\empty\else%
    \PYG@tok{#1}\expandafter\PYG@toks\fi}
\def\PYG@do#1{\PYG@bc{\PYG@tc{\PYG@ul{%
    \PYG@it{\PYG@bf{\PYG@ff{#1}}}}}}}
\def\PYG#1#2{\PYG@reset\PYG@toks#1+\relax+\PYG@do{#2}}

\expandafter\def\csname PYG@tok@ne\endcsname{\def\PYG@tc##1{\textcolor[rgb]{0.00,0.44,0.13}{##1}}}
\expandafter\def\csname PYG@tok@c1\endcsname{\let\PYG@it=\textit\def\PYG@tc##1{\textcolor[rgb]{0.25,0.50,0.56}{##1}}}
\expandafter\def\csname PYG@tok@gr\endcsname{\def\PYG@tc##1{\textcolor[rgb]{1.00,0.00,0.00}{##1}}}
\expandafter\def\csname PYG@tok@s1\endcsname{\def\PYG@tc##1{\textcolor[rgb]{0.25,0.44,0.63}{##1}}}
\expandafter\def\csname PYG@tok@se\endcsname{\let\PYG@bf=\textbf\def\PYG@tc##1{\textcolor[rgb]{0.25,0.44,0.63}{##1}}}
\expandafter\def\csname PYG@tok@kn\endcsname{\let\PYG@bf=\textbf\def\PYG@tc##1{\textcolor[rgb]{0.00,0.44,0.13}{##1}}}
\expandafter\def\csname PYG@tok@gh\endcsname{\let\PYG@bf=\textbf\def\PYG@tc##1{\textcolor[rgb]{0.00,0.00,0.50}{##1}}}
\expandafter\def\csname PYG@tok@cs\endcsname{\def\PYG@tc##1{\textcolor[rgb]{0.25,0.50,0.56}{##1}}\def\PYG@bc##1{\setlength{\fboxsep}{0pt}\colorbox[rgb]{1.00,0.94,0.94}{\strut ##1}}}
\expandafter\def\csname PYG@tok@nn\endcsname{\let\PYG@bf=\textbf\def\PYG@tc##1{\textcolor[rgb]{0.05,0.52,0.71}{##1}}}
\expandafter\def\csname PYG@tok@s2\endcsname{\def\PYG@tc##1{\textcolor[rgb]{0.25,0.44,0.63}{##1}}}
\expandafter\def\csname PYG@tok@mh\endcsname{\def\PYG@tc##1{\textcolor[rgb]{0.13,0.50,0.31}{##1}}}
\expandafter\def\csname PYG@tok@gs\endcsname{\let\PYG@bf=\textbf}
\expandafter\def\csname PYG@tok@kc\endcsname{\let\PYG@bf=\textbf\def\PYG@tc##1{\textcolor[rgb]{0.00,0.44,0.13}{##1}}}
\expandafter\def\csname PYG@tok@k\endcsname{\let\PYG@bf=\textbf\def\PYG@tc##1{\textcolor[rgb]{0.00,0.44,0.13}{##1}}}
\expandafter\def\csname PYG@tok@vc\endcsname{\def\PYG@tc##1{\textcolor[rgb]{0.73,0.38,0.84}{##1}}}
\expandafter\def\csname PYG@tok@nb\endcsname{\def\PYG@tc##1{\textcolor[rgb]{0.00,0.44,0.13}{##1}}}
\expandafter\def\csname PYG@tok@ss\endcsname{\def\PYG@tc##1{\textcolor[rgb]{0.32,0.47,0.09}{##1}}}
\expandafter\def\csname PYG@tok@sx\endcsname{\def\PYG@tc##1{\textcolor[rgb]{0.78,0.36,0.04}{##1}}}
\expandafter\def\csname PYG@tok@gt\endcsname{\def\PYG@tc##1{\textcolor[rgb]{0.00,0.27,0.87}{##1}}}
\expandafter\def\csname PYG@tok@sd\endcsname{\let\PYG@it=\textit\def\PYG@tc##1{\textcolor[rgb]{0.25,0.44,0.63}{##1}}}
\expandafter\def\csname PYG@tok@kp\endcsname{\def\PYG@tc##1{\textcolor[rgb]{0.00,0.44,0.13}{##1}}}
\expandafter\def\csname PYG@tok@mi\endcsname{\def\PYG@tc##1{\textcolor[rgb]{0.13,0.50,0.31}{##1}}}
\expandafter\def\csname PYG@tok@m\endcsname{\def\PYG@tc##1{\textcolor[rgb]{0.13,0.50,0.31}{##1}}}
\expandafter\def\csname PYG@tok@sh\endcsname{\def\PYG@tc##1{\textcolor[rgb]{0.25,0.44,0.63}{##1}}}
\expandafter\def\csname PYG@tok@mo\endcsname{\def\PYG@tc##1{\textcolor[rgb]{0.13,0.50,0.31}{##1}}}
\expandafter\def\csname PYG@tok@gu\endcsname{\let\PYG@bf=\textbf\def\PYG@tc##1{\textcolor[rgb]{0.50,0.00,0.50}{##1}}}
\expandafter\def\csname PYG@tok@go\endcsname{\def\PYG@tc##1{\textcolor[rgb]{0.20,0.20,0.20}{##1}}}
\expandafter\def\csname PYG@tok@vi\endcsname{\def\PYG@tc##1{\textcolor[rgb]{0.73,0.38,0.84}{##1}}}
\expandafter\def\csname PYG@tok@sr\endcsname{\def\PYG@tc##1{\textcolor[rgb]{0.14,0.33,0.53}{##1}}}
\expandafter\def\csname PYG@tok@nd\endcsname{\let\PYG@bf=\textbf\def\PYG@tc##1{\textcolor[rgb]{0.33,0.33,0.33}{##1}}}
\expandafter\def\csname PYG@tok@kt\endcsname{\def\PYG@tc##1{\textcolor[rgb]{0.56,0.13,0.00}{##1}}}
\expandafter\def\csname PYG@tok@c\endcsname{\let\PYG@it=\textit\def\PYG@tc##1{\textcolor[rgb]{0.25,0.50,0.56}{##1}}}
\expandafter\def\csname PYG@tok@gp\endcsname{\let\PYG@bf=\textbf\def\PYG@tc##1{\textcolor[rgb]{0.78,0.36,0.04}{##1}}}
\expandafter\def\csname PYG@tok@mb\endcsname{\def\PYG@tc##1{\textcolor[rgb]{0.13,0.50,0.31}{##1}}}
\expandafter\def\csname PYG@tok@ow\endcsname{\let\PYG@bf=\textbf\def\PYG@tc##1{\textcolor[rgb]{0.00,0.44,0.13}{##1}}}
\expandafter\def\csname PYG@tok@si\endcsname{\let\PYG@it=\textit\def\PYG@tc##1{\textcolor[rgb]{0.44,0.63,0.82}{##1}}}
\expandafter\def\csname PYG@tok@ge\endcsname{\let\PYG@it=\textit}
\expandafter\def\csname PYG@tok@gd\endcsname{\def\PYG@tc##1{\textcolor[rgb]{0.63,0.00,0.00}{##1}}}
\expandafter\def\csname PYG@tok@no\endcsname{\def\PYG@tc##1{\textcolor[rgb]{0.38,0.68,0.84}{##1}}}
\expandafter\def\csname PYG@tok@gi\endcsname{\def\PYG@tc##1{\textcolor[rgb]{0.00,0.63,0.00}{##1}}}
\expandafter\def\csname PYG@tok@bp\endcsname{\def\PYG@tc##1{\textcolor[rgb]{0.00,0.44,0.13}{##1}}}
\expandafter\def\csname PYG@tok@kd\endcsname{\let\PYG@bf=\textbf\def\PYG@tc##1{\textcolor[rgb]{0.00,0.44,0.13}{##1}}}
\expandafter\def\csname PYG@tok@w\endcsname{\def\PYG@tc##1{\textcolor[rgb]{0.73,0.73,0.73}{##1}}}
\expandafter\def\csname PYG@tok@na\endcsname{\def\PYG@tc##1{\textcolor[rgb]{0.25,0.44,0.63}{##1}}}
\expandafter\def\csname PYG@tok@sc\endcsname{\def\PYG@tc##1{\textcolor[rgb]{0.25,0.44,0.63}{##1}}}
\expandafter\def\csname PYG@tok@cm\endcsname{\let\PYG@it=\textit\def\PYG@tc##1{\textcolor[rgb]{0.25,0.50,0.56}{##1}}}
\expandafter\def\csname PYG@tok@o\endcsname{\def\PYG@tc##1{\textcolor[rgb]{0.40,0.40,0.40}{##1}}}
\expandafter\def\csname PYG@tok@vg\endcsname{\def\PYG@tc##1{\textcolor[rgb]{0.73,0.38,0.84}{##1}}}
\expandafter\def\csname PYG@tok@sb\endcsname{\def\PYG@tc##1{\textcolor[rgb]{0.25,0.44,0.63}{##1}}}
\expandafter\def\csname PYG@tok@cp\endcsname{\def\PYG@tc##1{\textcolor[rgb]{0.00,0.44,0.13}{##1}}}
\expandafter\def\csname PYG@tok@mf\endcsname{\def\PYG@tc##1{\textcolor[rgb]{0.13,0.50,0.31}{##1}}}
\expandafter\def\csname PYG@tok@err\endcsname{\def\PYG@bc##1{\setlength{\fboxsep}{0pt}\fcolorbox[rgb]{1.00,0.00,0.00}{1,1,1}{\strut ##1}}}
\expandafter\def\csname PYG@tok@nt\endcsname{\let\PYG@bf=\textbf\def\PYG@tc##1{\textcolor[rgb]{0.02,0.16,0.45}{##1}}}
\expandafter\def\csname PYG@tok@kr\endcsname{\let\PYG@bf=\textbf\def\PYG@tc##1{\textcolor[rgb]{0.00,0.44,0.13}{##1}}}
\expandafter\def\csname PYG@tok@s\endcsname{\def\PYG@tc##1{\textcolor[rgb]{0.25,0.44,0.63}{##1}}}
\expandafter\def\csname PYG@tok@nc\endcsname{\let\PYG@bf=\textbf\def\PYG@tc##1{\textcolor[rgb]{0.05,0.52,0.71}{##1}}}
\expandafter\def\csname PYG@tok@ni\endcsname{\let\PYG@bf=\textbf\def\PYG@tc##1{\textcolor[rgb]{0.84,0.33,0.22}{##1}}}
\expandafter\def\csname PYG@tok@nv\endcsname{\def\PYG@tc##1{\textcolor[rgb]{0.73,0.38,0.84}{##1}}}
\expandafter\def\csname PYG@tok@il\endcsname{\def\PYG@tc##1{\textcolor[rgb]{0.13,0.50,0.31}{##1}}}
\expandafter\def\csname PYG@tok@nf\endcsname{\def\PYG@tc##1{\textcolor[rgb]{0.02,0.16,0.49}{##1}}}
\expandafter\def\csname PYG@tok@nl\endcsname{\let\PYG@bf=\textbf\def\PYG@tc##1{\textcolor[rgb]{0.00,0.13,0.44}{##1}}}

\def\PYGZbs{\char`\\}
\def\PYGZus{\char`\_}
\def\PYGZob{\char`\{}
\def\PYGZcb{\char`\}}
\def\PYGZca{\char`\^}
\def\PYGZam{\char`\&}
\def\PYGZlt{\char`\<}
\def\PYGZgt{\char`\>}
\def\PYGZsh{\char`\#}
\def\PYGZpc{\char`\%}
\def\PYGZdl{\char`\$}
\def\PYGZhy{\char`\-}
\def\PYGZsq{\char`\'}
\def\PYGZdq{\char`\"}
\def\PYGZti{\char`\~}
% for compatibility with earlier versions
\def\PYGZat{@}
\def\PYGZlb{[}
\def\PYGZrb{]}
\makeatother

\renewcommand\PYGZsq{\textquotesingle}

\begin{document}

\maketitle
\tableofcontents
\phantomsection\label{index::doc}

\begin{description}
\item[{This website provides the documentation for the usage and extension of the INstrumentSETtings beam instrumentation toolbox.}] \leavevmode
This is not in the above paragraph?

\end{description}


\chapter{Contents}
\label{index:contents}\label{index:welcome-to-inset}\label{index:index-label}

\section{Beam}
\label{beam:module-beam}\label{beam:beam}\label{beam::doc}\label{beam:beam-label}\index{beam (module)}
This module defines the beam class

The module takes the following arguments

Property --- Key --- Value Type --- Remarks

Particle type --- par\_type --- String --- Ion type (p, U, Ar etc.)

Charge state --- charge\_state --- Integer --- Charge state

Atomic mass --- atomic\_mass ---  Integer --- 2 for Hydrogen

Particle energy --- kin\_energy --- Float --- Kinetic energy per nucleon

Particle number ---par\_num --- Integer --- Total number of particles (ions)

Distribution type --- d\_type --- String --- a for arbitrary, p for parabolic, g for gaussian and kv for KV distribution

X Distribution ---- x\_dist --- List of integers for `a', two Ints for parabolic and gaussian --- Phase space distribution in x plane

Y Distribution ---- y\_dist --- Same as X Dist. --- Phase space distribution in y plane

Z Distribution ---- z\_dist --- Same as X Dist. --- Phase space distribution in z/s plane
\begin{quote}

par\_type=None, charge\_state= None, atomic\_mass = None, par\_num= None, d\_type = `ggg', x\_dist = {[}0,5{]}, y\_dist ={[}0,5{]}, z\_dist ={[}0,100{]}
\end{quote}

The arguments can be passed in this order or by defining a dictionary or by calling a file where the parameters exist
\index{dynamicbeam (class in beam)}

\begin{fulllineitems}
\phantomsection\label{beam:beam.dynamicbeam}\pysiglinewithargsret{\strong{class }\code{beam.}\bfcode{dynamicbeam}}{\emph{*args}, \emph{**kwargs}}{}
This attributes are similar to a static beam, however, the beam energy, number of beam particles and beam structure is updated for several turns (depends on the length of list) and stored
\index{save() (beam.dynamicbeam method)}

\begin{fulllineitems}
\phantomsection\label{beam:beam.dynamicbeam.save}\pysiglinewithargsret{\bfcode{save}}{\emph{name\_of\_file}}{}
This function will save the beam object to an external file in the directory called ``defined\_beams'' in the source directory

\end{fulllineitems}


\end{fulllineitems}

\index{plot() (in module beam)}

\begin{fulllineitems}
\phantomsection\label{beam:beam.plot}\pysiglinewithargsret{\code{beam.}\bfcode{plot}}{}{}
This function will plot the profile of the beam in the mentioned axis

\end{fulllineitems}

\index{staticbeam (class in beam)}

\begin{fulllineitems}
\phantomsection\label{beam:beam.staticbeam}\pysiglinewithargsret{\strong{class }\code{beam.}\bfcode{staticbeam}}{\emph{*args}, \emph{**kwargs}}{}
Beam class defines the static beam object

It creates a beam object instance the parameters in a special order are specified, or simply by passing a beam dictionary

A save keyword `s' can be used to save the beam object in a file, which can be loaded later
\index{load() (beam.staticbeam method)}

\begin{fulllineitems}
\phantomsection\label{beam:beam.staticbeam.load}\pysiglinewithargsret{\bfcode{load}}{\emph{name\_of\_file}}{}
This function will load the beam object from the specified file in the directory called ``defined\_beams'' in the source directory

\end{fulllineitems}

\index{save() (beam.staticbeam method)}

\begin{fulllineitems}
\phantomsection\label{beam:beam.staticbeam.save}\pysiglinewithargsret{\bfcode{save}}{\emph{name\_of\_file}}{}
This function will save the beam object to an external file in the directory called ``defined\_beams'' in the source directory

\end{fulllineitems}


\end{fulllineitems}

\index{structure() (in module beam)}

\begin{fulllineitems}
\phantomsection\label{beam:beam.structure}\pysiglinewithargsret{\code{beam.}\bfcode{structure}}{}{}
This function defines the structure of the beam based on the beam parameters

\end{fulllineitems}



\section{Indices and tables}
\label{beam:indices-and-tables}\begin{itemize}
\item {} 
\DUspan{xref,std,std-ref}{genindex}

\item {} 
\DUspan{xref,std,std-ref}{modindex}

\item {} 
\DUspan{xref,std,std-ref}{search}

\item {} 
{\hyperref[index::doc]{\emph{\emph{Home}}}}

\item {} 
{\hyperref[index:index-label]{\emph{\DUspan{}{Table of contents}}}}

\item {} 
{\hyperref[beam:beam-label]{\emph{\DUspan{}{Table of page}}}}

\end{itemize}


\section{Machine}
\label{machine:machine}\label{machine::doc}\label{machine:module-machine}\index{machine (module)}
This module defines the beam object

The module takes the following arguments

Property --- Key --- Value type --- Description

Circumference --- circumference --- Float --- Circumference of the machine

Compaction factor --- com\_fact --- Float --- Momentum compaction factor

Set tune --- set\_tune --- List of Float --- Horizontal and vertical tune

Set Chromaticity --- set\_chro --- List of float --- Horizontal and vertical chromaticity
\index{dynamicmachine (class in machine)}

\begin{fulllineitems}
\phantomsection\label{machine:machine.dynamicmachine}\pysiglinewithargsret{\strong{class }\code{machine.}\bfcode{dynamicmachine}}{\emph{*args}, \emph{**kwargs}}{}
This attributes are similar to a static beam, however, the beam energy, number of beam particles and beam structure is updated for several turns (depends on the length of list) and stored
\index{save() (machine.dynamicmachine method)}

\begin{fulllineitems}
\phantomsection\label{machine:machine.dynamicmachine.save}\pysiglinewithargsret{\bfcode{save}}{\emph{name\_of\_file}}{}
This function will save the beam object to an external file in the directory called ``defined\_beams'' in the source directory

\end{fulllineitems}


\end{fulllineitems}

\index{staticmachine (class in machine)}

\begin{fulllineitems}
\phantomsection\label{machine:machine.staticmachine}\pysiglinewithargsret{\strong{class }\code{machine.}\bfcode{staticmachine}}{\emph{*args}, \emph{**kwargs}}{}
The machine class defines all the machine parameters
\index{save() (machine.staticmachine method)}

\begin{fulllineitems}
\phantomsection\label{machine:machine.staticmachine.save}\pysiglinewithargsret{\bfcode{save}}{\emph{name\_of\_file}}{}
This function will save the beam object to an external file in the directory called ``defined\_beams'' in the source directory

\end{fulllineitems}


\end{fulllineitems}



\subsection{Indices and tables}
\label{machine:indices-and-tables}\begin{itemize}
\item {} 
\DUspan{xref,std,std-ref}{genindex}

\item {} 
\DUspan{xref,std,std-ref}{modindex}

\item {} 
\DUspan{xref,std,std-ref}{search}

\item {} 
{\hyperref[index::doc]{\emph{\emph{Home}}}}

\item {} 
{\hyperref[index:index-label]{\emph{\DUspan{}{Table of contents}}}}

\item {} 
{\hyperref[beam:beam-label]{\emph{\DUspan{}{Table of page}}}}

\end{itemize}

Use the tutorial here to learn about Sphinx: \phantomsection\label{machine:id1}{\hyperref[machine:sphinxdoc]{\emph{{[}SPHINXDOC{]}}}}.


\section{Device Modules}
\label{device_modules:device-modules}\label{device_modules::doc}
Each Diagnostic sensor description and settings are documented here.


\subsection{Current Transformers}
\label{device_modules:current-transformers}
Generic Transformer object
\phantomsection\label{device_modules:module-TrafoModule}\index{TrafoModule (module)}
Generic Trafo module takes beam and machine object and returns the TrafoOut

The module takes the following arguments

Beam --- Beam object fully specifying the beam

Machine --- Accelerator setting object

TrafoType (Optional) --- Specific transformer types to define exact Trafo behaviour
\index{generictrafo (class in TrafoModule)}

\begin{fulllineitems}
\phantomsection\label{device_modules:TrafoModule.generictrafo}\pysiglinewithargsret{\strong{class }\code{TrafoModule.}\bfcode{generictrafo}}{\emph{*args}, \emph{**kwargs}}{}
The Generic trafo class defines all the generic trafo parameters
\index{save() (TrafoModule.generictrafo method)}

\begin{fulllineitems}
\phantomsection\label{device_modules:TrafoModule.generictrafo.save}\pysiglinewithargsret{\bfcode{save}}{\emph{name\_of\_file}}{}
This function will save the beam object to an external file in the directory called ``defined\_beams'' in the source directory

\end{fulllineitems}


\end{fulllineitems}



\subsection{Add functions from Python library}
\label{device_modules:add-functions-from-python-library}
\href{https://docs.python.org/library/io.html\#io.open}{\code{io.open()}}


\subsection{Indices and tables}
\label{device_modules:indices-and-tables}\begin{itemize}
\item {} 
\DUspan{xref,std,std-ref}{genindex}

\item {} 
\DUspan{xref,std,std-ref}{modindex}

\item {} 
\DUspan{xref,std,std-ref}{search}

\item {} 
{\hyperref[index::doc]{\emph{\emph{Home}}}}

\item {} 
{\hyperref[index:index-label]{\emph{\DUspan{}{Table of contents}}}}

\item {} 
{\hyperref[beam:beam-label]{\emph{\DUspan{}{Table of page}}}}

\end{itemize}


\section{Common Modules}
\label{common_modules:common-modules}\label{common_modules::doc}
The common modules consist of electronics, optics systems cables etc. They are described and documented here.


\subsection{Amplifiers and Attenuators}
\label{common_modules:amplifiers-and-attenuators}
Generic amplifier and attenuator definition
\phantomsection\label{common_modules:module-AmpAttModule}\index{AmpAttModule (module)}
Generic Amplifier Module

The module takes the following arguments

Amplification --- The amplification/attenuation in (dB)

Noise figure --- Accelerator setting object

Input Noise --- When the input is open or terminated (in nV/sqrt(Hz))

AmplifierType (Optional) --- Specific amplifier implementation
\index{genericAmpAtt (class in AmpAttModule)}

\begin{fulllineitems}
\phantomsection\label{common_modules:AmpAttModule.genericAmpAtt}\pysiglinewithargsret{\strong{class }\code{AmpAttModule.}\bfcode{genericAmpAtt}}{\emph{*args}, \emph{**kwargs}}{}
The Generic trafo class defines all the generic trafo parameters
\index{save() (AmpAttModule.genericAmpAtt method)}

\begin{fulllineitems}
\phantomsection\label{common_modules:AmpAttModule.genericAmpAtt.save}\pysiglinewithargsret{\bfcode{save}}{\emph{name\_of\_file}}{}
This function will save the beam object to an external file in the directory called ``defined\_beams'' in the source directory

\end{fulllineitems}


\end{fulllineitems}



\subsection{Lenses}
\label{common_modules:lenses}

\subsection{Indices and tables}
\label{common_modules:indices-and-tables}\begin{itemize}
\item {} 
\DUspan{xref,std,std-ref}{genindex}

\item {} 
\DUspan{xref,std,std-ref}{modindex}

\item {} 
\DUspan{xref,std,std-ref}{search}

\item {} 
{\hyperref[index::doc]{\emph{\emph{Home}}}}

\item {} 
{\hyperref[index:index-label]{\emph{\DUspan{}{Table of contents}}}}

\item {} 
{\hyperref[beam:beam-label]{\emph{\DUspan{}{Table of page}}}}

\end{itemize}


\chapter{Indices and tables}
\label{index:indices-and-tables}\begin{itemize}
\item {} 
\DUspan{xref,std,std-ref}{genindex}

\item {} 
\DUspan{xref,std,std-ref}{modindex}

\item {} 
\DUspan{xref,std,std-ref}{search}

\end{itemize}

Use the tutorial here to learn about Sphinx: \phantomsection\label{index:id1}{\hyperref[index:sphinxdoc]{\emph{{[}SPHINXDOC{]}}}}.

\begin{thebibliography}{SPHINXDOC}
\bibitem[SPHINXDOC]{SPHINXDOC}{\phantomsection\label{index:sphinxdoc} 
This is Sphinx doc documentation --\textgreater{} \href{http://sphinx-doc.org/latest/tutorial.html}{http://sphinx-doc.org/latest/tutorial.html}.
}
\end{thebibliography}


\renewcommand{\indexname}{Python Module Index}
\begin{theindex}
\def\bigletter#1{{\Large\sffamily#1}\nopagebreak\vspace{1mm}}
\bigletter{a}
\item {\texttt{AmpAttModule}}, \pageref{common_modules:module-AmpAttModule}
\indexspace
\bigletter{b}
\item {\texttt{beam}}, \pageref{beam:module-beam}
\indexspace
\bigletter{m}
\item {\texttt{machine}}, \pageref{machine:module-machine}
\indexspace
\bigletter{t}
\item {\texttt{TrafoModule}}, \pageref{device_modules:module-TrafoModule}
\end{theindex}

\renewcommand{\indexname}{Index}
\printindex
\end{document}
